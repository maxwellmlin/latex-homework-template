% 
% Homework Details
% 
\newcommand{\hmwkTitle}{Homework\ \#11}
\newcommand{\hmwkDueDate}{April 26, 2023}
\newcommand{\hmwkClass}{MATH 222}
\newcommand{\hmwkClassTime}{Section 2}
\newcommand{\hmwkClassInstructor}{Professor Levine}
\newcommand{\hmwkAuthorName}{\textbf{Maxwell Lin}}

% 
% Document class
% 
\documentclass{article}

% 
% Packages
% 
\usepackage{fancyhdr}
\usepackage{extramarks}
\usepackage{amsmath}
\usepackage{amssymb}
\usepackage{amsthm}
\usepackage{amsfonts}
\usepackage{tikz}
\usepackage[plain]{algorithm}
\usepackage{algpseudocode}
\usepackage{enumerate}
\usepackage{subcaption}
\usepackage{pgfplots}
\pgfplotsset{compat=1.18}

\usetikzlibrary{automata, positioning}

%
% Basic Document Settings
%
\topmargin=-0.45in
\evensidemargin=0in
\oddsidemargin=0in
\textwidth=6.5in
\textheight=9.0in
\headsep=0.25in

\linespread{1.1}

% Headers/Footers
\pagestyle{fancy}
\lhead{\hmwkAuthorName}
\chead{\hmwkClass\ —\ \hmwkTitle}
\rhead{\firstxmark}
\lfoot{\lastxmark}
\cfoot{\thepage}

\renewcommand\headrulewidth{0.4pt}
\renewcommand\footrulewidth{0.4pt}

\setlength\parindent{0pt}

% Black square QED toggle
% \renewcommand{\qedsymbol}{$\blacksquare$}

%
% Create Problem Sections
%

\newcommand{\enterProblemHeader}[1]{
    \nobreak\extramarks{}{Problem \arabic{#1} continued on next page\ldots}\nobreak{}
    \nobreak\extramarks{Problem \arabic{#1} (continued)}{Problem \arabic{#1} continued on next page\ldots}\nobreak{}
}

\newcommand{\exitProblemHeader}[1]{
    \nobreak\extramarks{Problem \arabic{#1} (continued)}{Problem \arabic{#1} continued on next page\ldots}\nobreak{}
    \stepcounter{#1}
    \nobreak\extramarks{Problem \arabic{#1}}{}\nobreak{}
}

\setcounter{secnumdepth}{0}
\newcounter{partCounter}
\newcounter{homeworkProblemCounter}
\setcounter{homeworkProblemCounter}{1}
\nobreak\extramarks{Problem \arabic{homeworkProblemCounter}}{}\nobreak{}

%
% Homework Problem Environment
%
% This environment takes an optional argument. When given, it will adjust the
% problem counter. This is useful for when the problems given for your
% assignment aren't sequential. See the last 3 problems of this template for an
% example.
%
\newenvironment{homeworkProblem}[1][-1]{
    \ifnum#1>0
        \setcounter{homeworkProblemCounter}{#1}
    \fi
    \section{Problem \arabic{homeworkProblemCounter}}
    \setcounter{partCounter}{1}
    \enterProblemHeader{homeworkProblemCounter}
}{
    \exitProblemHeader{homeworkProblemCounter}
}

%
% Title Page
%

\title{
    \vspace{-.5in}
    \textmd{\textbf{\hmwkClass\ —\ \hmwkTitle}}\\
    \normalsize\vspace{0.1in}\small{Due\ \hmwkDueDate}\\
}

\author{\hmwkAuthorName}
\date{}

\renewcommand{\part}[1]{\textbf{\large Part \Alph{partCounter}}\stepcounter{partCounter}\\}

%
% Various Helper Commands
%

% Useful for algorithms
\newcommand{\alg}[1]{\textsc{\bfseries \footnotesize #1}}

% For derivatives
\newcommand{\deriv}[1]{\frac{\mathrm{d}}{\mathrm{d}x} (#1)}

% For partial derivatives
\newcommand{\pderiv}[2]{\frac{\partial #1}{\partial #2}}

% Integral dx
\newcommand{\dx}{\mathrm{d}x}

% Alias for the Solution section header
\newcommand{\solution}{\textbf{\large Solution}}

% Probability commands: Expectation, Variance, Covariance, Bias
\newcommand{\E}{\mathrm{E}}
\newcommand{\Var}{\mathrm{Var}}
\newcommand{\Cov}{\mathrm{Cov}}
\newcommand{\Bias}{\mathrm{Bias}}

% Set of reals
\newcommand{\R}{\mathbb{R}}

% Set of integers
\newcommand{\Z}{\mathbb{Z}}

% Right arrow
\newcommand{\ra}{\rightarrow}

% Differentiation operators
\newcommand{\D}{\mathbf{D}}
\newcommand{\Hess}{\mathbf{H}}

% Reset equation numbering
\newcommand{\reset}{\setcounter{equation}{0}}

% Insert picture
\newcommand{\pic}[1]{\includegraphics[width=\textwidth]{#1}}

% Divergence operator
\renewcommand{\div}{\operatorname{div}}
\newcommand{\curl}{\operatorname{curl}}

% Matrix spacing
\makeatletter
\renewcommand*\env@matrix[1][\arraystretch]{%
  \edef\arraystretch{#1}%
  \hskip -\arraycolsep
  \let\@ifnextchar\new@ifnextchar
  \array{*\c@MaxMatrixCols c}}
\makeatother

\begin{document}

\maketitle

\begin{homeworkProblem}
    Section 7.2: \#4
    
    Evaluate each of the following line integrals:
    \\(a) $\int_{\mathbf{c}} x d y-y d x, \quad \mathbf{c}(t)=(\cos t, \sin t)$ $0 \leq t \leq 2 \pi$
    \\(b) $\int_{\mathbf{c}} x d x+y d y, \quad \mathbf{c}(t)=(\cos \pi t, \sin \pi t)$, $0 \leq t \leq 2$
    \\(c) $\int_{\mathbf{c}} y z d x+x z d y+x y d z$, where c consists of straight-line segments joining $(1,0,0)$ to $(0,1,0)$ to $(0,0,1)$
    \\(d) $\int_{\mathbf{c}} x^2 d x-x y d y+d z$, where $\mathbf{c}$ is the parabola $z=x^2, y=0$ from $(-1,0,1)$ to $(1,0,1)$
    \\

    \solution

    \begin{enumerate}[(a)]
        \item We have
        \begin{align*}
            x&=\cos t & dx &= -\sin t \, dt \\
            y &= \sin t & dy &= \cos t \, dt.
        \end{align*}

        Thus,
        \begin{align*}
            \int_{\mathbf{c}} x d y-y d x &= \int_0^{2\pi} \cos^2(t)+\sin^2(t) \, dt \\
            &= 2\pi.
        \end{align*}

        \item We have
        \begin{align*}
            \int_{\mathbf{c}} x d x+y d y &= \int_0^2 -\pi \cos(\pi t)\sin(\pi t) + \pi\sin(\pi t)\cos(\pi t) \, dt \\
            &= 0.
        \end{align*}

        \item We have
        \begin{align*}
            \int_{\mathbf{c}} y z d x+x z d y+x y d z = \int_{c_1} y z d x+x z d y+x y d z + \int_{c_2} y z d x+x z d y+x y d z
        \end{align*}
        where $c_1$ parameterizes the straight-line segment joining $(1,0,0)$ to $(0,1,0)$ and $c_2$ parameterizes the straight-line-segment joining $(0,1,0)$ to $(0,0,1)$.

        We obtain
        \begin{align*}
            c_1 &= (1-t, t, 0) \qquad t \in [0,1] \\ 
            c_2 &= (0, 1-t, t) \qquad t \in [0,1].
        \end{align*}

        Thus,
        \begin{align*}
            \int_{\mathbf{c}} y z d x+x z d y+x y d z &= \int_{c_1} 0+0+0 + \int_{c_2} 0+0+0 \\
            &= 0.
        \end{align*}

        \item 
        We parameterize $c$ with
        \begin{align*}
            c(t) = (t,0,t^2) \qquad t \in [-1,1].
        \end{align*}

        Thus,
        \begin{align*}
            \int_{\mathbf{c}} x^2 d x-x y d y+d z &= \int_{-1}^1t^2+2t \, dt \\
            &= \left[\frac{t^3}{3}+t^2\right]_{-1}^1 \\
            &= \frac{2}{3}.
        \end{align*}
    \end{enumerate}
\end{homeworkProblem}

\begin{homeworkProblem}
    Section 7.2: \#12
    
    Suppose $\mathbf{c}_1$ and $\mathbf{c}_2$ are two paths with the same endpoints and $\mathbf{F}$ is a vector field. Show that $\int_{\mathbf{c}_1} \mathbf{F} \cdot d \mathbf{s}=\int_{\mathbf{c}_2} \mathbf{F} \cdot d \mathbf{s}$ is equivalent to $\int_c \mathbf{F} \cdot d \mathbf{s}=0$, where $C$ is the closed curve obtained by first moving along $\mathbf{c}_1$ and then moving along $\mathbf{c}_2$ in the opposite direction.
    \\

    \solution
    
    Since the oriented curve $C$ can be obtained by moving along the curve parameterized by $c_1$ then moving along the curve parameterized by $c_2$ in the opposite orientation,
    \begin{align*}
        \int_C F \cdot ds = \int_{c_1} F \cdot ds - \int_{c_2} F \cdot ds = 0.
    \end{align*}
    This is equivalent to
    \begin{equation*}
        \int_{\mathbf{c}_1} \mathbf{F} \cdot d \mathbf{s}=\int_{\mathbf{c}_2} \mathbf{F} \cdot d \mathbf{s}.
    \end{equation*}
    
\end{homeworkProblem}

\begin{homeworkProblem}
    Section 7.3: \#8

    Match the following parametrizations to the surfaces shown in the figures.
    \\(a) $\Phi(u, v)=(u \cos v, u \sin v, 4-u \cos v-u \sin v)$; $u \in[0,1], v \in[0,2 \pi]$
    \\(b) $\Phi(u, v)=\left(u \cos v, u \sin v, 4-u^2\right)$
    \\(c) $\Phi(u, v)=\left(u, v, \frac{1}{3}(12-8 u-3 v)\right)$
    \\(d) $\Phi(u, v)=\left(\left(u^2+6 u+11\right) \cos v\right.$, $\left.u,\left(u^2+6 u+11\right) \sin v\right)$
    
    \includegraphics[scale=.5]{11-3}

    \solution

    \begin{enumerate}[(a)]
        \item (i)
        \item (iii)
        \item (ii)
        \item (iv)
    \end{enumerate}
\end{homeworkProblem}

\begin{homeworkProblem}
    Section 7.3: \#23

    The image of the parametrization
    \begin{equation*}
    \begin{aligned}
    \Phi(u, v) & =(x(u, v), y(u, v), z(u, v)) \\
    & =((R+r \cos u) \cos v,(R+r \cos u) \sin v, r \sin u)
    \end{aligned}
    \end{equation*}
    with $0 \leq u, v \leq 2 \pi, 0<r<1, R>1$ parametrizes a torus (or doughnut) $S$.
    \\(a) Show that all points in the image $(x, y, z)$ satisfy:
    \begin{equation*}
    \left(\sqrt{x^2+y^2}-R\right)^2+z^2=r^2 .
    \end{equation*}
    \\(b) Show that the image surface is regular at all points.
    \\

    \solution

    \begin{enumerate}[(a)]
        \item We have
        \begin{align*}
            \left(\sqrt{x^2+y^2}-R\right)^2+z^2&=\left(\sqrt{(R+r\cos u)^2\cos^2 v + (R+r\cos u)^2\sin^2v}-R\right)^2+r^2\sin^2u \\
            &= r^2\cos^2u + r^2\sin^2u \\
            &= r^2
        \end{align*}
        as required.

        \item 
        To show that the image surface is regular at all points, we must show that $T_u \times T_v \ne 0$.

        We have
        \begin{align*}
            T_u=\begin{bmatrix}
                -r\sin u\cos v \\
                -r\sin u \sin v \\
                r \cos u
            \end{bmatrix} \qquad
            T_v=\begin{bmatrix}
                -(R+r\cos u) \sin v\\
                (R+r\cos u) \cos v\\
                0
            \end{bmatrix}
        \end{align*}
        \begin{equation*}
            T_u \times T_v = -r(R+r\cos u) \begin{bmatrix}
                \cos u \cos v \\
                \cos u \sin v \\
                \sin u.
            \end{bmatrix}
        \end{equation*}

        By assumption, $r \ne 0$. Additionally, $R+r\cos u \ne 0$ since $-1 < r\cos u < 1$ and $R>1$. Therefore, we must prove that 
        \begin{equation*}
            A = \begin{bmatrix}
                \cos u \cos v \\
                \cos u \sin v \\
                \sin u
            \end{bmatrix} \ne 0.
        \end{equation*}
        For the sake of contradiction, assume that $A=0$. Then, $\sin u = 0$ which means $u=k\pi$ for $k \in \Z^+ \cup \{0\}$. We also must have that 
        \begin{gather*}
            \cos u \cos v = \cos u \sin v = 0 \\
            \cos v = \sin v = 0 \qquad \text{since $\cos u \ne 0$}.
        \end{gather*}
        However, this equation has no solutions. Therefore, $A\ne0$ and the image surface must be regular at all points.
    \end{enumerate}
\end{homeworkProblem}

\begin{homeworkProblem}
    Section 7.4: \#1
    
    Find the surface area of the unit sphere $S$ represented parametrically by $\boldsymbol{\Phi}: D \rightarrow S \subset \mathbb{R}^3$, where $D$ is the rectangle $0 \leq \theta \leq 2 \pi, 0 \leq \phi \leq \pi$ and $\boldsymbol{\Phi}$ is given by the equations
    \begin{equation*}
    x=\cos \theta \sin \phi, \quad y=\sin \theta \sin \phi, \quad z=\cos \phi
    \end{equation*}
    Note that we can represent the entire sphere parametrically, but we cannot represent it in the form $z=f(x, y)$
    \\

    \solution

    The formula for surface area is
    \begin{equation*}
        \iint_D \|T_\theta \times T_\phi\| \, dA.
    \end{equation*}

    We have
    \begin{gather*}
        T_\theta = \begin{bmatrix}
            -\sin\phi\sin\theta \\
            \sin\phi\cos\theta \\
            0
        \end{bmatrix} \qquad
        T_\phi = \begin{bmatrix}
            \cos\theta\cos\phi \\
            \sin\theta\cos\phi \\
            -\sin\phi
        \end{bmatrix} \\
        T_\theta \times T_\phi = \begin{bmatrix}
            -\sin^2\phi \cos\theta \\
            -\sin^2\phi\sin\theta \\
            -\sin\phi\cos\phi
        \end{bmatrix}
    \end{gather*}
    \begin{align*}
        \|T_\theta \times T_\phi\| &= \sqrt{\sin^4\phi \cos^2\theta + \sin^4\phi\sin^2\theta + \sin^2\phi\cos^2\phi} \\
        &=\sqrt{\sin^4\phi + \sin^2\phi\cos^2\phi}\\
        &=\sqrt{\sin^2\phi} \\
        &=\sin\phi \qquad \text{since $\phi \in [0, \pi]$}.
    \end{align*}

    Therefore,
    \begin{align*}
        \iint_D \|T_\theta \times T_\phi\| \, dA &= \int_0^\pi\int_0^{2\pi}(\sin\phi) \, d\theta \, d\phi \\
        &= 4\pi.
    \end{align*}
\end{homeworkProblem}

\begin{homeworkProblem}
    Section 7.4: \#4

    The torus $T$ can be represented parametrically by the function $\boldsymbol{\Phi}: D \rightarrow \mathbb{R}^3$, where $\boldsymbol{\Phi}$ is given by the coordinate functions $x=(R+\cos \phi) \cos \theta$ $y=(R+\cos \phi) \sin \theta, z=\sin \phi ; D$ is the rectangle $[0,2 \pi] \times[0,2 \pi]$, that is, $0 \leq \theta \leq 2 \pi, 0 \leq \phi \leq 2 \pi$; and $R>1$ is fixed (see Figure 7.4.6). Show that $A(T)=(2 \pi)^2 R$, first by using formula (3) and then by using formula (6).
    \\

    \solution

    Formula (3) is
    \begin{equation*}
    A(S)=\iint_D \sqrt{\left[\frac{\partial(x, y)}{\partial(\theta, \phi)}\right]^2+\left[\frac{\partial(y, z)}{\partial(\theta, \phi)}\right]^2+\left[\frac{\partial(x, z)}{\partial(\theta, \phi)}\right]^2} \, d\theta \, d\phi.
    \end{equation*}

    We compute as follows
    \begin{align*}
        \pderiv{(x,y)}{(\theta,\phi)} &= \begin{vmatrix}
            -(R+\cos\phi)\sin\theta & -\sin\phi\cos\theta \\
            (R+\cos\phi)\cos\theta & -\sin\phi\sin\theta
        \end{vmatrix}\\
        &= (R+\cos\phi)\sin\phi
    \end{align*}
    \begin{align*}
        \pderiv{(y,z)}{(\theta,\phi)} &= \begin{vmatrix}
            (R+\cos\phi)\cos\theta & -\sin\phi\sin\theta \\
            0 & \cos\phi
        \end{vmatrix}\\
        &= (R+\cos\phi)\cos\theta\cos\phi
    \end{align*}
        \begin{align*}
        \pderiv{(x,y)}{(\theta,\phi)} &= \begin{vmatrix}
            -(R+\cos\phi)\sin\theta & -\sin\phi\cos\theta \\
            0 & \cos\phi
        \end{vmatrix}\\
        &= -(R+\cos\phi)\sin\theta\cos\phi
    \end{align*}
    \begin{equation*}
        \sqrt{\left[\frac{\partial(x, y)}{\partial(\theta, \phi)}\right]^2+\left[\frac{\partial(y, z)}{\partial(\theta, \phi)}\right]^2+\left[\frac{\partial(x, z)}{\partial(\theta, \phi)}\right]^2} = R+\cos\phi.
    \end{equation*}

    Thus, we have
    \begin{align*}
        A(s) &= \int_0^{2\pi} \int_0^{2\pi} (R+\cos\phi) \, d\phi \, d\theta \\
        &= \int_0^{2\pi} 2\pi R \, d\theta \\
        &= (2\pi)^2R
    \end{align*}
    as required.
    \\
    
    Formula (6) is
    \begin{equation*}
    A=2 \pi \int_a^b\left(|x| \sqrt{1+\left[f^{\prime}(x)\right]^2}\right) d x
    \end{equation*}

    We have
    \begin{equation*}
        f(x) = \sqrt{1-(x-R)^2}
    \end{equation*}
    which is the graph of the upper half cross-section of the torus we wish to revolve about the y-axis.

    We compute
    \begin{equation*}
        f'(x)= \frac{-(x-R)}{\sqrt{1-(x-R)^2}}
    \end{equation*}
    and
    \begin{align*}
        \sqrt{1+[f'(x)]^2} &= \sqrt{1+\frac{(x-R)^2}{1-(x-R)^2}} \\
        &= \frac{1}{\sqrt{1-(x-R)^2}}.
    \end{align*}

    Thus, we have
    \begin{align*}
        2 \pi \int_a^b\left(|x| \sqrt{1+\left[f^{\prime}(x)\right]^2}\right) d x &= 2\pi \int_{R-1}^{R+1}\frac{x}{\sqrt{1-(x-R)^2}} \, dx \qquad \text{$|x|=x$ since $R>1$} \\
        &= 2\pi \left[\int_{R-1}^{R+1}\frac{x-R}{\sqrt{1-(x-R)^2}} \, dx + \int_{R-1}^{R+1}\frac{R}{\sqrt{1-(x-R)^2}} \, dx\right]
    \end{align*}

    We compute the first integral
    \begin{align*}
        \int_{R-1}^{R+1}\frac{x-R}{\sqrt{1-(x-R)^2}} \, dx &= \int_0^0 \frac{du}{2\sqrt{u}} \qquad \text{substitute $u=1-(x-R)^2$} \\
        &= 0.
    \end{align*}

    We compute the second integral
    \begin{align*}
        \int_{R-1}^{R+1}\frac{R}{\sqrt{1-(x-R)^2}} \, dx &= [R \arcsin(x-R)]_{R-1}^{R+1} \\
        &= \pi R.
    \end{align*}

    Thus, 
    \begin{equation*}
        2 \pi \int_a^b\left(|x| \sqrt{1+\left[f^{\prime}(x)\right]^2}\right) d x = 2\pi^2 R.
    \end{equation*}

    Multiplying by 2 (since we can only graph half of the torus cross-section), we obtain $4\pi^2R$ as required.
    
    

    
\end{homeworkProblem}

\begin{homeworkProblem}
    Section 7.5: \#9

    Evaluate $\iint_S z d S$, where $S$ is the upper hemisphere of radius $a$, that is, the set of $(x, y, z)$ with $z=\sqrt{a^2-x^2-y^2}$
    \\

    \solution
    
    Using spherical coordinates, we obtain the surface parametrization
    \begin{equation*}
        \Phi(\theta, \phi)=(a\sin\phi\cos\theta,a\sin\phi\sin\theta,a\cos\phi)
    \end{equation*}
    where $\phi \in [0,\pi/2]$ and $\theta \in [0,2\pi]$.
    \\
    
    We have
    \begin{gather*}
        T_\theta = \begin{bmatrix}
            -a\sin\phi\sin\theta \\
            a\sin\phi\cos\theta\\
            0
        \end{bmatrix} \qquad
        T_\phi = \begin{bmatrix}
            a\cos\phi\cos\theta\\
            a\cos\phi\sin\theta\\
            -a\sin\phi
        \end{bmatrix}\\
        T_\theta \times T_\phi = \begin{bmatrix}
            -a^2\sin^2\phi\cos\theta\\
            -a^2\sin^2\phi\sin\theta\\
            -a^2\cos\phi\sin\phi
        \end{bmatrix}
    \end{gather*}
    \begin{align*}
        \|T_\theta \times T_\phi\| = a^2\sin\phi.
    \end{align*}

    Thus, we obtain
    \begin{align*}
        \iint_S z d S &= \int_0^{2\pi} \int_0^{\pi/2} (a\cos\phi)(a^2\sin\phi) \, d\phi \, d\theta\\
        &= \int_0^{2\pi} \frac{a^3}{2} \, d\theta \\
        &= \pi a^3.
    \end{align*}
\end{homeworkProblem}

\begin{homeworkProblem}
    Section 7.5: \#12

    Verify that in spherical coordinates, on a sphere of radius $R$,
    \begin{equation*}
    \left\|\mathbf{T}_\phi \times \mathbf{T}_\theta\right\| \, d \phi \, d \theta=R^2 \sin \phi \, d \phi \, d \theta
    \end{equation*}

    \solution
    
    We have the surface parametrization
    \begin{equation*}
        \Phi(\theta, \phi)=(R\sin\phi\cos\theta,R\sin\phi\sin\theta,R\cos\phi)
    \end{equation*}
    where $\phi \in [0,\pi]$ and $\theta \in [0,2\pi]$.
    \\
    
    We compute
    \begin{gather*}
        T_\theta = \begin{bmatrix}
            -R\sin\phi\sin\theta \\
            R\sin\phi\cos\theta\\
            0
        \end{bmatrix} \qquad
        T_\phi = \begin{bmatrix}
            R\cos\phi\cos\theta\\
            R\cos\phi\sin\theta\\
            -R\sin\phi
        \end{bmatrix}\\
        T_\theta \times T_\phi = \begin{bmatrix}
            -R^2\sin^2\phi\cos\theta\\
            -R^2\sin^2\phi\sin\theta\\
            -R^2\cos\phi\sin\phi
        \end{bmatrix}
    \end{gather*}
    \begin{align*}
        \|T_\theta \times T_\phi\| &= \sqrt{(-R^2\sin^2\phi\cos\theta)^2+(-R^2\sin^2\phi\sin\theta)^2 + (-R^2\cos\phi\sin\phi)^2} \\
        &= \sqrt{R^4\sin^2\phi[\sin^2\phi(\cos^2\theta+\sin^2\theta)+\cos^2\phi]} \\
        &= \sqrt{R^4\sin^2\phi}\\
        &= R^2\sin\phi \qquad \text{since $R^2\sin\phi\ge0$}
    \end{align*}
    as required.
\end{homeworkProblem}

\begin{homeworkProblem}
    Section 7.6: \#2
    
    Evaluate the surface integral
    \begin{equation*}
    \iint_S \mathbf{F} \cdot d \mathbf{S}
    \end{equation*}
    where $\mathbf{F}(x, y, z)=x \mathbf{i}+y \mathbf{j}+z^2 \mathbf{k}$ and $S$ is the surface parameterized by $\Phi(u, v)=(2 \sin u, 3 \cos u, v)$, with $0 \leq u \leq 2 \pi$ and $0 \leq v \leq 1$
    \\

    \solution

    We have
    \begin{gather*}
        T_u = \begin{bmatrix}
            2\cos u\\
            -3\sin u\\
            0
        \end{bmatrix} \qquad
        T_v = \begin{bmatrix}
            0\\0\\1
        \end{bmatrix}\\
        T_u \times T_v = \begin{bmatrix}
            -3\sin u\\
            -2\cos u\\
            0
        \end{bmatrix}\\
        (T_u \times T_v) \cdot F =\begin{bmatrix}
            -3\sin u\\
            -2\cos u\\
            0
        \end{bmatrix} \cdot \begin{bmatrix}
            2\sin u\\
            3\cos u\\
            v^2
        \end{bmatrix} = -6.
    \end{gather*}

    Thus, we obtain
    \begin{align*}
        \iint_S \mathbf{F} \cdot d \mathbf{S} &= \int_0^{2\pi}\int_0^1 -6\,dv\,du \\
        &= -12\pi.
    \end{align*}
\end{homeworkProblem}

\begin{homeworkProblem}
    Section 7.6: \#7

    Let $S$ be the closed surface that consists of the hemisphere $x^2+y^2+z^2=1, z \geq 0$, and its base $x^2+y^2 \leq 1, z=0$. Let $\mathbf{E}$ be the electric field defined by $\mathbf{E}(x, y, z)=2 x \mathbf{i}+2 y \mathbf{j}+2 z \mathbf{k}$. Find the electric flux across $S$. (HINT: Break $S$ into two pieces $S_1$ and $S_2$ and evaluate $\iint_{S_1} \mathbf{E} \cdot d \mathbf{S}$ and $\iint_{S_2} \mathbf{E} \cdot d \mathbf{S}$ separately.)
    \\

    \solution
    
    The electric flux across $S$ is
    \begin{equation*}
        \iint_S E \cdot \, dS = \iint_{S_1} E \cdot \, dS + \iint_{S_2} E \cdot \, dS
    \end{equation*}
    where $S_1$ is the hemisphere and $S_2$ is the base, both with unit normals that face outwards. We parametrize $S_1$ with
    \begin{equation*}
        \Phi_1(\theta,\phi)=(\sin\phi\cos\theta, \sin\phi\sin\theta,\cos\phi)
    \end{equation*}
    where $\phi\in[0,\pi/2]$ and $\theta\in[0,2\pi]$.
    \\
    
    From Problem 7, we know that
    \begin{equation*}
        T_\phi \times T_\theta = \begin{bmatrix}
            \sin^2\phi\cos\theta\\
            \sin^2\phi\sin\theta\\
            \cos\phi\sin\phi
        \end{bmatrix}.
    \end{equation*}

    Thus,
    \begin{align*}
        E \cdot (T_\phi \times T_\theta) &= \begin{bmatrix}
            2\sin\phi\cos\theta \\
            2\sin\phi\sin\theta\\
            2\cos\phi
        \end{bmatrix}
        \cdot
        \begin{bmatrix}
            \sin^2\phi\cos\theta\\
            \sin^2\phi\sin\theta\\
            \cos\phi\sin\phi
        \end{bmatrix} \\
        &= 2\sin\phi.
    \end{align*}

    \begin{align*}
        \iint_{S_1} E \cdot \, dS &= \int_0^{2\pi} \int_0^{\pi/2} (2\sin\phi) \, d\phi \, d\theta \\
        &= 4\pi.
    \end{align*}

    Now we parameterize $S_2$ by
    \begin{equation*}
        \Phi_2(r,\theta) = (r\cos\theta, r\sin\theta, 0)
    \end{equation*}
    where $r\in[0,1]$ and $\theta\in[0,2\pi]$. We obtain
    \begin{gather*}
        T_r = \begin{bmatrix}
            \cos\theta\\
            \sin\theta\\
            0
        \end{bmatrix} \qquad
        T_\theta = \begin{bmatrix}
            -r\sin\theta\\
            r\cos\theta\\
            0
        \end{bmatrix} \\
        T_\theta \times T_r = \begin{bmatrix}
            0 \\ 0 \\ -r
        \end{bmatrix}.
    \end{gather*}

    Thus,
    \begin{align*}
        E \cdot (T_\theta \times T_r) &= \begin{bmatrix}
            2r\cos\theta\\
            2r\sin\theta\\
            0
        \end{bmatrix}
        \cdot
        \begin{bmatrix}
            0 \\ 0 \\ -r
        \end{bmatrix} \\
        &= 0.
    \end{align*}

    Therefore,
    \begin{equation*}
        \iint_{S_2} E \cdot \, dS = 0
    \end{equation*}
    and 
    \begin{equation*}
        \iint_S E \cdot \, dS = 4\pi.
    \end{equation*}
\end{homeworkProblem}

\end{document}
